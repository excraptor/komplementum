Bevezetés

Hálók -> Hálószerűen rendezett halmazok -> részbenrendezés -> relációk -> ekvivalenciarelációk/osztályozások
(legnagyobb) alsó korlát, (legkisebb) felső korlát - ez egyértelmű
pár példa
teljes háló
részháló
N5, M3

Relációk

Az $A$ halmaz önmagával vett Descartes-szorzatának részhalmazait relációknak nevezzük. Ez tulajdonképpen elempárokat jelent,
ahol mindkét elem az $A$ halmazból kerül ki. 

Példa: a $\leq$ az $A = \{1,2,3\}$ halmaz önmagával vett Descartes-szorzatán relációként értelmezve a következő részhalmaz lesz: 
$\rho = \{(1,1), (1,2), (1,3), (2,2), (2,3), (3,3)\}$. Minden elem pontosan azzal van relációban, amelyik elemnél kisebb, vagy egyenlő.

Relációk tulajdonságai

Egy $\alpha \in A \times A$ reláció esetén azt mondjuk, hogy $\alpha$

\begin{itemize}

    \item reflexív, ha bármely x \in A-ra (x,x) \in \alpha
\end{itemize}

Ekvivalenciarelációk/osztályozások

Részbenrendezés

Hálószerűen rendezett halmaz

(legnagyobb) alsó korlát, (legkisebb) felső korlát (+egyértelműség)

Hálók

Teljes hálók

Részhálók

N5, M3 nem disztributívak
